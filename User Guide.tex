%% LyX 2.0.2 created this file.  For more info, see http://www.lyx.org/.
%% Do not edit unless you really know what you are doing.
\documentclass[english]{article}
\usepackage[T1]{fontenc}
\usepackage[latin9]{inputenc}
\usepackage{listings}
\setlength{\parskip}{\smallskipamount}
\setlength{\parindent}{0pt}
\usepackage{babel}
\begin{document}

\title{\LaTeX{} For Code Documentation User Guide}


\author{Garrett Berg, cloudformdesign.com}

\maketitle

\section{Use Lyx}

Lyx is a simple, easy to use software. In this section I will explain
how to set it up, how to set the keybindings to be more sane (for
someone used to LibreOffice at least), and how to set it up to be
able to export your code documentation correctly.


\section{Exporting Code}

If you installed my configuration settings, then you don't need to
do anything for this. If you haven't though, you can set it up by
going to Tools -> Preferences -> File Handling -> File Formats. Create
a new one named ``Codedoc''. Make the short name whatever you want
and the extension ``html.'' Also change the viewer to your webbrowser
(mine is ``firefox'')

Now go to \textbf{Converters }and immediately press the drop-box on
the bottom left. Change it to \textbf{\LaTeX{} (plain)} and the ``To
format'' change to your newly created \textbf{Codedoc}. Finally,
in the converted box, copy paste the below:

\begin{lstlisting}
NEED THE PYTHON COMMAND LINE
\end{lstlisting}



\end{document}
