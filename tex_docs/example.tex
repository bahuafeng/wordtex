%% LyX 2.0.2 created this file.  For more info, see http://www.lyx.org/.
%% Do not edit unless you really know what you are doing.
\documentclass[english]{article}
\usepackage[T1]{fontenc}
\usepackage[latin9]{inputenc}
\usepackage{listings}
\setlength{\parskip}{\smallskipamount}
\setlength{\parindent}{0pt}
\usepackage{babel}
\usepackage{array}
\usepackage{multirow}
\usepackage[unicode=true]
 {hyperref}
\usepackage{breakurl}

\makeatletter

%%%%%%%%%%%%%%%%%%%%%%%%%%%%%% LyX specific LaTeX commands.
%% Because html converters don't know tabularnewline
\providecommand{\tabularnewline}{\\}

\makeatother

\begin{document}



\section*{The Cloud Toolbox}

I recently released the code for my \href{https://github.com/cloudformdesign/cloudtb}{cloudtb}.
The definition I have of a toolbox is this: functions that should
be included in the python standard library, but aren't. This is a
general library which I intend to use across all my projects in the
future. This library is released as FULLY open source (MIT License).
I think I will make most of the modules fully community commons soon
so that anyone can feel free to edit or change them (and steal from
them) if they wish.

I am currently messing around with Git, but what I want is to give
anyone the ability to create their own personal branch. \textbf{I
want you to use this toolbox as YOUR toolbox}. In addition, if you
already have a ``toolbox,'' then feel free to just create a branch
and copy - paste the functions you find valuable into it.

One of the major features of this toolbox that I am implementing is
the requirement that everything be fully documented in \LaTeX{}. This
is a simple document formating tool that can be exported to html,
pdf, and even wordpress! This blog post is written in \LaTeX{} using
the Lyx text editor. I will soon release the code that I used to create
this blog post as well.


\section*{Memory Management}


\subsection*{Iterators}

The best method of memory management already has tons of internal
support in python: iterators. However, the libraries provided in the
standard library fall a little short in a couple of areas, primarily
\textit{adding iterators}~and doing \textit{look-aheads. }To address
these problems I developed two classes, the \textbf{biter} class and
the \textbf{soliditer }class. These classes can be found in \href{https://github.com/cloudformdesign/cloudtb/commit/405f5b903a5270249cdb5e52f1454b9bdce8416a}{the iterators class}.
These two classes extend iteration to these nearly essential capabilities.
I hope to write soliditer in c so that it's speed limitations can
be solved.


\subsection*{Auto-Swap (Harddrive) Data}

As far as I know, this is a new idea (and certainly hasn't been implemented).
The idea is simple: if you don't use a variable very often it should
be stored in the harddrive. My first few attempts have run into a
few current problems with implementing the basic wrapper class (as
can be seen in \href{http://stackoverflow.com/questions/19202997/python-general-wrapper-class-for-any-data}{this stackoverflow question}),
but I have hope that this can eventually become a reality.

Auto-swap data could significantly help with memory management, particularily
for large, persistent applications like GUI's. The basic implementation
is as follows:

\begin{lstlisting}
# want to extend my iterator 
myiter = biter(myiter) 
bi2 = biter(range(100)) # use xrange in python2 
myiter = myiter + bi2 # ... other stuff
# now I only want every 5th element 
myiter = myiter[::5]
# ... other stuff 

# Now I just want the 20th element (note: removes elements before.         
# if you want to keep these elements use soliditer) 
el20 = myiter[20]
\end{lstlisting}



\section*{Inserting a Table}

\begin{tabular}{|c|c}
\hline 
This is a table & It modifies itself automatically\tabularnewline
\hline 
\hline 
No, it did not. Interesting but not all that important I guess... & col 1, row 0\tabularnewline
\hline 
0,1 & col1, r1\tabularnewline
\hline 
0,2 & \multirow{1}{*}{1,2}\tabularnewline
\hline 
0,3 \& that was an and & 1,3\tabularnewline
\hline 
\end{tabular}

I'm going to do some special characters now

From 1-0

!@\#\$\%\textasciicircum{}\&{*}()

\begin{tabular}{|c|c|}
\hline 
exc & !\tabularnewline
\hline 
\hline 
at & @\tabularnewline
\hline 
pound & \#\tabularnewline
\hline 
money & \$\tabularnewline
\hline 
perc & \%\tabularnewline
\hline 
carrot & \textasciicircum{}\tabularnewline
\hline 
and & \&\tabularnewline
\hline 
star & {*}\tabularnewline
\hline 
parans & ()\tabularnewline
\hline 
brackets & {[}{]}\tabularnewline
\hline 
curly & \{\}\tabularnewline
\hline 
backslash & \textbackslash{}\tabularnewline
\hline 
fslash & /\tabularnewline
\hline 
question & ?\tabularnewline
\hline 
quote 1 & '\tabularnewline
\hline 
quote 2 & ``\tabularnewline
\hline 
tilde & \textasciitilde{}\tabularnewline
\hline 
 & \tabularnewline
\hline 
 & \tabularnewline
\hline 
 & \tabularnewline
\hline 
\end{tabular}
\end{document}
