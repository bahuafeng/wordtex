%% LyX 2.0.2 created this file.  For more info, see http://www.lyx.org/.
%% Do not edit unless you really know what you are doing.
\documentclass[english]{article}
\usepackage[T1]{fontenc}
\usepackage[latin9]{inputenc}
\usepackage{listings}
\setlength{\parskip}{\smallskipamount}
\setlength{\parindent}{0pt}
\usepackage{babel}
\usepackage[unicode=true]
 {hyperref}
\usepackage{breakurl}
\begin{document}

\title{Wordtex -- The Latex Publishing Software for Python}


\author{Original Copyright Garrett Berg \href{http://cloudformdesign.com}{cloudformdesign.com}}

\maketitle
This project and file are Licensed under the GNU Geneal Public License
v3 or greater. For more information see \href{http://www.gnu.org/licenses/}{http://www.gnu.org/licenses/}


\section{About}

Wordtex is an attempt to be a full feature Latex conversion software
to html (specifically for wordpress html, but aims to be general)
that aims to replace the README.txt files in most open source projects.
For easy to read documentation viewing, you should be looking at the
``developer.html'' version of this file. This file was originally
created with the extremely easy to use \LaTeX{} editor, Lyx, and converted
using itself.


\subsection{Latex Advantages}

There are several reasons that using Latex is superior for code documentation
than the current methods I have found.
\begin{enumerate}
\item Can automatically output to a more readable format, such as making
lists like this one. The hope is to integrate it with the publish
module in cloudtb to automatically create the documentation files
every version.
\item Plays nicely with revision control allowing easy tracking of changes
to documentation.
\item The output (html) also plays nicely with revision control, so you
can even track the output so new developers can start reading without
executing a script.

\begin{enumerate}
\item The hope is to eventually integrate the output (html) to automatically
publish on a wiki or website.
\end{enumerate}
\item Is a fully extensible markup language, allowing you to do almost anything
you want (provided this module has been extended to handle it!)
\item Can eventually provide extensible features such as automatc importing
of python documentation. The hope is that code such as \textit{\textbackslash{}python\{docs,module,function\}
}will automatically import the doc function of that python module
during conversion.
\item Instead of wiki's, which certainly have many uses, this text automatically
ships with the package and is more accessible to offline developers.
In addition, having the developer documentation reside next to the
code \textit{just makes sense}.
\end{enumerate}

\section{Code Overview}

Word\TeX{} was written in a highly object oriented manner in order
to be easily extend-able. All formatting options are kept in \textbf{formatting.py}
and it should be easy to add any option you want to do with only a
minimal knowledge of python. See the below documentation on \textbf{wp\_formatting.py}
for more info.


\subsection{wordtex.py}

This file is the main entry point and mostly handles user input. No
classes or functions are defined in here that are used in the module
as a whole.


\subsection{publish.py}

This is a publish file under the same effort as that in cloudtb. It
currently only adjusts the license, but the hope is to integrate it
fully to automatically publish documentation like this.


\subsection{wp\_formatting.py}


\subsubsection{Overview}

This is the primary formatting module, and provides easy extensibility
of Latex function. Latex format types are split into five broad categories
\begin{enumerate}
\item \textit{\textbackslash{}begin\{...\} \textbackslash{}end -- }The whole
textblock is outlined by these headers
\item \textit{\textbackslash{}if... \textbackslash{}fi }-- The text starts
and ends with these statements
\item Text attributes of the form \textit{\textbackslash{}attr\{your text\}}~
\item Line items of the form \textit{\textbackslash{}lineitem} that end
with the newline character.
\end{enumerate}
Each item is of the following format

\begin{lstlisting}
format_list = [
[item1,tp(...)]
[item2,tp(...)]
]
\end{lstlisting}


the ``items'' in the above act as both labels (to be used in debugging)
and in the respective templates for each list. You can see how your
new item will be formatted by looking at the templates in the \textbf{build\_dict
}call immediately below your list. 

\textbf{IMPORTANT: }All items are in regular expression format! To
read more about regular expressions, go here (need link) 


\subsubsection{More Detail: The Internals of formatting.}

In order for Word\TeX{} directly uses the formatting module (primarily
the \textbf{every\_dict\_formatting} dictionary) to directly construct
it's TexPart objects. \textbf{TexPart.update}, which is called every
time text is initialized, goes through every formatting regular expression
for it's \textit{text\_data} and tries to convert them to TexPart
objects. This happens through two functions:
\begin{enumerate}
\item \textbf{texlib.get\_text\_data} is called with a list of text\_objects
-- which is just a list containing concatenated strings and TexPart
objects (there will never be two strings next to eachother, this is
ensured with the \textbf{reform\_text}~function). \textbf{get\_text\_data}
splits up this list using\textbf{ texpart\_constructor.match\_re }and
then goes through this list, splitting it up into interesting sections
-- primarily the start (value 2), body (value 1) and end (value 3)
of each text part. It ignores TexPart objects, as they have already
been processed.
\item It then automatically passes this data onto \textbf{convert\_inout},
which uses the start, body, and end properties provided by the above
to construct the objects by using copy.copy() on the text\_part constructor.
\end{enumerate}

\subsection{textlib.py}


\subsubsection{Overview}

This is the main module for wordtex. The primary class is \textbf{TexPart},
which is what handles all the formating. This class is intended to
make it extremely easy to extend the formatting options available
in the Word\TeX{} package. 

The conversion process directly uses this module as goes as follows:
\begin{enumerate}
\item An outside function (i.e. \textbf{process\_document }in this module)
initializes the document class and creates the first TexPart. 
\item This automatically (through it's \textbf{init\_text} member function)
recursively creates the internal TexPart objects that reside in it's
text data.
\item Each TexPart has it's formatting options defined in \textbf{wp\_formatting.py},
after everything has been imported, the \textbf{.format} method is
called for the highest level TexPart (normally document), which recursively
formats all the data.

\begin{enumerate}
\item Another method is to simply call the \textbf{.get\_wp\_text} method,
which automatically calls the \textbf{.format}~method.
\end{enumerate}
\end{enumerate}

\subsubsection{Additional notes:}
\begin{enumerate}
\item All data in TexPart is recoverable, as long as the \textbf{format}~function
hasn't been called. This allows parent TexParts to override the automatic
conversion process and convert what WordTex \textit{thinks }are objects
back into regular text (can be done through \textbf{.reset\_text}
or \textbf{.get\_original\_text}). This makes creating special formatting
very straightforward, as you can signal to let your custom \textbf{call\_first}
and \textbf{call\_last }options handle all the text.
\item All formating options in \textbf{wp\_formatting.py} are defined in
TexParts. When converting, the class is automatically copied. See
\textbf{convert\_processed} inside of \textbf{convert\_inout} for
an example on how to do this.
\end{enumerate}

\section{Helping with this Project}

Git repository 


\subsection{Project Goals}

Word\TeX{} aims to integrate with \textbf{publish.py }found in \href{http://Cloud Toolbox}{https://github.com/cloudformdesign/cloudtb}
in order to become a defacto-standard for use with docmenting python
code. The advantages of latex for this purpose are outlined in Section
1.

Word\TeX{} is a part of the \textbf{cloudtb}~project which aims to
provide a uniform, integrated toolbox for the open source community
to publish their useful python code.

Word\TeX{} also aims to make writing posts for blogs a lot easier,
in fact it started because I (Garrett) was sick of conventional blogging
software and just wanted something that \textit{just worked}. As I
continued to use Latex and the program developed by Luca Trevisan,
I realized that Latex could be a lot more. Trying to update his code
to fit my needs proved an act of frustration, however his code gave
me inspiration as well as a somewhat-usable base for Word\TeX{}.


\subsection{TODO List}
\begin{itemize}
\item Add math functionality back in -- Lucas's original code was designed
to handle math. This project has different goals, but should still
provide this funcationality
\item \ifpy (print("hello world")\fi <--(may be invisible) Standard Latex
viewers don't do anything with this
\item \ifpy doc('module', 'function')\fi This is how I want to pull documentation
\item I was edited
\item Add \textbackslash{}python\{fuction, {*}args\} methods -- These can
be used to insert text directly from python macros.

\begin{itemize}
\item related -- directly insert pydocs of functions
\item related -- copy to-do list from the repository to-do list, so that
lists like this can be automated in the future!
\item related -- define document python variables to be used in these functions
\end{itemize}
\item Tables -- tables are not yet supported. I did not like Luca's original
implementation so I am searching for a different one
\item integrate fully with publish.py\end{itemize}

\end{document}
